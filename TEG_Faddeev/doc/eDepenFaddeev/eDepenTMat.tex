\documentclass[11pt]{article}

\usepackage{bbm}
\usepackage{geometry}
\usepackage{graphicx}
\usepackage{amsmath}
\usepackage{float}
\usepackage{listings}
\usepackage{indentfirst}
\usepackage{braket}
\usepackage{authblk}
\usepackage{hyperref}
\setlength{\parindent}{2em}
\geometry{left=2.5cm,right=2.5cm,top=2.0cm,bottom=2.0cm}
% About the author
\renewcommand*{\Authand}{, }
\renewcommand\Affilfont{\itshape\small}
\lstset{language=C}

\date{}

\title{Variation of Two-Body Potential Does Not Change the Form of the Faddeev Equation}
\author[1]{Zeyuan Ye}
\setcounter{Maxaffil}{0}

\begin{document}
  \maketitle

  Notations in this paper are consistent with the ones in \textit{Elster's
    note}. We start from the Faddeev equation in the momentum space (equation
  (3.33) in \textit{Elster's note}),

  \begin{equation}
    \braket{\boldsymbol{p q}| \psi} = \int{\int{d^3p'' d^3q''}} \int{\int{d^3p' d^3q'}} \braket{\boldsymbol{pq}|G_0t|\boldsymbol{p'q'}} \braket{\boldsymbol{p'q'}|P|\boldsymbol{p''q''}} \braket{\boldsymbol{p'' q''}|\psi}
  \end{equation}
  Stripe the t part. Notice that t only depends on the two-body-CM energy, $E
  - \frac{3}{4m}q^2$, and diagonalize the third particle's momentum $q$,
  \begin{equation}
    \braket{\boldsymbol{pq}|t(E)|\boldsymbol{p'q'}} = \delta(\boldsymbol{q - q'}) \braket{\boldsymbol{p}|t(E - \frac{3}{4m}q^2)|\boldsymbol{p'}}
    \end{equation}
    The only thing that an alteration of potential can change is the value of
    $\braket{\boldsymbol{p}|t(E - \frac{3}{4m}q^2)|\boldsymbol{p'}}$ which comes
    from the solution of LSE, yet the form of calling t matrix remains the same.
    So the form of Faddeev equation remains the same.

\end{document}